Pseudonymization is a reversible de-identification process, in which personal data is converted to a \emph{pseudonym}, i.e. a unique identifier that can only be linked again to the personal data with certain restrictions (authorized re-identification in the single projects in SFB289). The main purpose of pseudonymization is to securely separate experimental from personal data. In clinical research, reversibility is typically required due to potential incidental findings.

A proper pseudonymization protocol can motivate the relaxation, to a certain degree, of data controllers’ legal obligations (if properly applied) \cite{pseudonym}, i.e. less effort is required to maintain data security and privacy protection throughout the whole life cycle of experimental data and its derivatives (from acquisition to sharing, publishing and archiving).

Issues of privacy protection are part of a rapidly changing 'landscape' shaped by ongoing digitalization efforts (in general and specifically in medical research) and by the recent developments in the corresponding regulations (most importantly the GDPR \cite{gdpr}). Recent developments in  this filed render some of the 'traditional' pseudonymization methods, which have been typically used so far in medical research (e.g. the frequently used sequential numbering of participants), increasingly “outdated”, with significant safety concerns (e.g. managing access to documents linking the personal data to the IDs).

Here we propose a software tool, called ALIIAS, which implements a 2-factor authenticated, de-centralized, encryption-based deterministic pseudonymization technique to transform personal data to a pseudonym.
The 'full version' of the pseudonym (long ID) allows for complete re-identification (given a dedicated secret digital key, owned by the 'pseudonymization entity', i.e. the individual research site). The software also provides a 'human-readable' (9 characters) and scannable (barcode)  short ID, which is easy to link to the long ID and is compatible with most experimental procedures.

ALIIAS is equipped with integration to LimeSurvey \cite{limesurvey}, an open-source web application for digital, web-accessible surveys and questionnaires.

\par\noindent\rule{\textwidth\color{pniblue}}{0.4pt}

\textbf{Highlights: }
\begin{itemize}
    \item pseudonymization happens as a first step of the experiment, so that all succeeding steps already work with anonymized data;
    \item pseudonyms are deterministic, i.e. reproducible in case of multiple measurements;
    \item pseudonyms are guaranteed to be unique, even for multi-center experiments;
    \item pseudonymization can be performed at multiple computers/sites simultaneously
    \item pseudonymization happens after a two-factor authentication, and requires a hardware key;
    \item no central administration of the link between pseudonym and personal data is needed
    \item the "pseudonymization secret" is restricted to a single point: to the USB hardware keys.
\end{itemize}




