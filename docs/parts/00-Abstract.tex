Pseudonymization is a reversible de-identification process, in which personal data is converted to a pseudonym (ID) that can only be linked again to the personal data by the “pseudonymization entity” (i.e. the single projects in SFB289). The main purpose of pseudonymization is to securely separate experimental from personal data. In clinical research, reversibility is required typically in case of incidental findings.

A proper pseudonymisation protocol can motivate the relaxation, to a certain degree, of data controllers’ legal obligations (if properly applied), i.e. spare a significant amount of efforts put into data security and privacy, when storing, sharing and publishing experimental data \cite{pseudonym}.

Issues of privacy protection constitute a rapidly changing landscape shaped by ongoing digitalization efforts (in general and especially in medical research) and the recent developments in the corresponding regulations (e.g. GDPR). Recent developments render the "traditional" pseudonymization methods typically used in medical reasearch (e.g. the frequently used sequential numbering of participants) increasingly “outdated”, with significant safety concerns (e.g. vulnerability stemming from storing and regularly updating a document linking the personal data to the IDs, or possibility of adversarial re-identification based on the order of data in the pseudonymized dataset).

Here we propose a software tool, called PseudoID, which implements a de-centralized, encryption-based pseudonymization technique which transforms personal data to a pseudonym.
The full of the pseudonym (long ID) allows for complete re-identification (given a dedicated secret digital key, owned by the 'pseudonymization entity', i.e. the individual research site). The software also provides a 'human-readable' short ID (8 characters or barcode), which is easy to link to the long ID and is compatible with most experimental procedures.

PseudoID is equipped with integration to LimeSurvey \cite{limesurvey}, an open-source web application for digital, web-accessible surveys and questionnaires.

\par\noindent\rule{\textwidth\color{pniblue}}{0.4pt}

\textbf{With PseudoID: }
\begin{itemize}
    \item pseudonymization happens as a first step of the experiment, allowing for all succeeding steps to be anonymized;
    \item pseudonyms are deterministic, i.e. reproducible in case of multiple measurements;
    \item pseudonyms are guaranteed (at a pre-defined error level) to be unique for distributed, multi-center experiments of any size;
    \item pseudonymization can be performed at multiple computers/sites simultaneously
    \item pseudonymization happens after a two-factor authentication, and requires a hardware key (or USB stick);
    \item no central administration of the link between pseudonym and personal data is needed
    \item the "pseudonymization secret" is reduced to the (arbitrary number of) USB hardware keys.
\end{itemize}



